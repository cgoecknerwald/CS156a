% Claire Goeckner-Wald

\documentclass[10pt,letter]{article}
\usepackage{amsmath}
\usepackage{amssymb}
\usepackage{graphicx}
\usepackage{setspace}
\onehalfspacing
\usepackage{fullpage}

\newcommand{\sign}{\text{sign}}

\begin{document}
\title{Problem Set 5}
\author{Claire Goeckner-Wald}
\maketitle 

%%%%%%%%%%%%%%%%%%%%%%%%%%%%%%%%%%%%%%%%%%%%%%%%%%%%%%%%%%%%%%%%%%%%%%%%%%%%%%%%
\section*{Linear Regression Error} 

\paragraph{1.} [c] 100

	\begin{align*}
	\sigma &= 0.1 \\
	d &= 8 \\
	E_{in} &> 0.008 \\
	\mathbb E_{\mathcal D}[E_{\text {in}}(\textbf w_{\text {lin}})] &> \sigma^2 (1 - \frac{d+1}{N}) \\
	\frac{\mathbb E_{\mathcal D}[E_{\text {in}}(\textbf w_{\text {lin}})]}{\sigma^2} &> 1 - \frac{d+1}{N} \\
	\frac{d+1}{N} &> 1 - \frac{\mathbb E_{\mathcal D}[E_{\text {in}}(\textbf w_{\text {lin}})]}{\sigma^2} \\
	N &> \frac{d+1}{1 - \frac{\mathbb E_{\mathcal D}[E_{\text {in}}(\textbf w_{\text {lin}})]}{\sigma^2}} \\
	N &> \frac{8+1}{1 - \frac{0.008}{0.1^2}} \\
	N &> 45
	\end{align*}

%%%%%%%%%%%%%%%%%%%%%%%%%%%%%%%%%%%%%%%%%%%%%%%%%%%%%%%%%%%%%%%%%%%%%%%%%%%%%%%%
\section*{Nonlinear Transforms}

\paragraph{2.} [D] $\widetilde {w }_1 < 0, \widetilde {w }_2 > 0$

	\begin{align*}
	\Phi(1, x_1, x_2) &= (1, x_1^2, x_2^2) \\
	\sign(\widetilde {\textbf w }^T \cdot \widetilde x) &=  \sign(\widetilde {w }_0 \cdot 1 + \widetilde { w }_1 \cdot \widetilde x_1 + \widetilde { w }_2 \cdot \widetilde x_2)\\
	\end{align*}

	As $\widetilde x_1$ increases, the function should be more likely to return a negative number. Thus, $\widetilde {w }_1$ should be negative as well. As $ \widetilde x_2$ increases, the function should be more likely to return a positive number. Thus, $\widetilde { w }_2$ should be positive as well.

\paragraph{3.} [D] 20

	The $d_{vc} \leq d + 1$. Here, $d = 15$, so $d_{vc} \leq 16$. Since none of the additional values are linear combinations of $x_1$ and $x_2$, then they are each new dimensions. 

%%%%%%%%%%%%%%%%%%%%%%%%%%%%%%%%%%%%%%%%%%%%%%%%%%%%%%%%%%%%%%%%%%%%%%%%%%%%%%%%
\section*{Gradient Descent}

\paragraph{4.} [D] $\widetilde{w}_1 < 0; \widetilde{w}_2 > 0$

	\begin{align*}
	E(u,v) &= (ue^v - 2ve^{-u})^2 \\
	\frac{\partial E}{\partial u} &= \frac{\partial (ue^v - 2ve^{-u})^2}{\partial u} \\
	&= 2(ue^v - 2ve^{-u})\cdot(e^v + 2ve^{-u}) \\
	\end{align*}

\paragraph{5.} [D] 10

	See attached code.

\paragraph{6.} [E] (0.045, 0.024)

	See attached code. 

\paragraph{7.} [A] $10^{-1}$

	See attached code. 

%%%%%%%%%%%%%%%%%%%%%%%%%%%%%%%%%%%%%%%%%%%%%%%%%%%%%%%%%%%%%%%%%%%%%%%%%%%%%%%%
\section*{Logistic Regression}

\paragraph{8.} [E] 0.125

	See attached code. 

\paragraph{9.} [A] 350

	See attached code. 

%%%%%%%%%%%%%%%%%%%%%%%%%%%%%%%%%%%%%%%%%%%%%%%%%%%%%%%%%%%%%%%%%%%%%%%%%%%%%%%%
\section*{PLA as SGD}

\paragraph{10.} [B] $-y_n \textbf w^{T} \textbf x_n$

	PLA and SGD are similar because they use one point at a time to decrease the error. Although the minimization of error may cause small increases in error on other points, the average error reduces over time.

\end{document}

