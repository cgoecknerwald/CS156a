% Claire Goeckner-Wald

\documentclass[10pt,letter]{article}
\usepackage{amsmath}
\usepackage{amssymb}
\usepackage{graphicx}
\usepackage{setspace}
\onehalfspacing
\usepackage{fullpage}

\begin{document}
\title{Problem Set 4}
\author{Claire Goeckner-Wald}
\maketitle 

%%%%%%%%%%%%%%%%%%%%%%%%%%%%%%%%%%%%%%%%%%%%%%%%%%%%%%%%%%%%%%%%%%%%%%%%%%%%%%%%
\section*{Generalization Error}

\paragraph{1.} D. 460,000

	\begin{align*}
	\epsilon &= \sqrt{\frac{8}{N} \cdot \ln{\frac{4m_\mathcal{H}(2N)}{\delta}} } \\
	0.95 &\geq 1 - \delta \\
	\delta &\leq 0.05 \\
	\epsilon &= 0.05 \\
	d_{vc} &= 10 \\
	m_\mathcal{H}(2N) &= (2N)^{d_{vc}} = (2N)^{10} \\
	0.05 &= \sqrt{\frac{8}{N} \cdot \ln{\frac{4(2N)^{10}}{0.05}} } \\
	(0.05)^2 &= \frac{8}{N} \cdot \ln{\frac{4(2N)^{10}}{0.05}}  \\
	\frac{N(0.05)^2}{8} &= \ln{\frac{4(2N)^{10}}{0.05}}  \\
	N &= -0.323, 0.323, 452957 \\
	\end{align*}

\paragraph{2.} D. Devroye

	\begin{align*}
	\delta &= 0.05 \\
	d_{vc} &= 50 \\
	m_\mathcal{H}(N) &= N^{d_{vc}} = N^{50} \\
	\end{align*}

	\subparagraph{[a]}
		\begin{align*}
		\epsilon &\leq \sqrt{\frac{8}{N} \cdot \ln{\frac{4m_\mathcal{H}(2N)}{\delta}} } \\
		\epsilon &\leq \sqrt{\frac{8}{N} \cdot \ln{\frac{4(2N)^{50}}{0.05}} } \\
		\epsilon &\leq 0.632
		\end{align*}
	\subparagraph{[b]}
		\begin{align*}
		\epsilon &\leq \sqrt{\frac{2\ln(2Nm_\mathcal{H}(N))}{N}} + \sqrt{\frac{2}{N} \ln{\frac{1}{\delta}}} + \frac{1}{N} \\
		\epsilon &\leq \sqrt{\frac{2\ln(2N N^{50} )}{N}} + \sqrt{\frac{2}{N} \ln{\frac{1}{0.05}}} + \frac{1}{N} \\
		\epsilon &\leq 0.331
		\end{align*}
	\subparagraph{[c]}
		\begin{align*}
		\epsilon &\leq \sqrt{\frac{1}{N}(2\epsilon + \ln\frac{6m_\mathcal{H}(2N)}{\delta} )} \\
		\epsilon &\leq \sqrt{\frac{1}{N}(2\epsilon + \ln\frac{6 (2N)^{50} }{0.05} )} \\
		100 \epsilon &\leq \sqrt{2 \epsilon +499.962} \\
		\epsilon &\leq 0.223
		\end{align*}
	\subparagraph{[d]} 
		\begin{align*}
		\epsilon &\leq \sqrt{\frac{1}{2N} (4\epsilon(1 +\epsilon) + \ln\frac{4m_\mathcal{H}(N^2)}{\delta} ) } \\
		\epsilon &\leq \sqrt{\frac{1}{2N} (4\epsilon(1 +\epsilon) + \ln\frac{4 (N^2)^{50} }{0.05} ) } \\
		\epsilon &\leq \sqrt{(4 \epsilon (\epsilon+1)+925.416)/(100 \sqrt{2}} \\
		\epsilon &\leq 0.215 \\
		\end{align*}

\paragraph{3.} C. Parrondo and Van den Broek

	\subparagraph{[a]}
		\begin{align*}
		\epsilon &\leq \sqrt{\frac{8}{N} \cdot \ln{\frac{4(2N)^{50}}{0.05}} } \\
		\epsilon &\leq 13.82
		\end{align*}
	\subparagraph{[b]}
		\begin{align*}
		\epsilon &\leq \sqrt{\frac{2\ln(2N N^{50} )}{N}} + \sqrt{\frac{2}{N} \ln{\frac{1}{0.05}}} + \frac{1}{N} \\
		\epsilon &\leq 7.05
		\end{align*}
	\subparagraph{[c]}
		\begin{align*}
		\epsilon &\leq \sqrt{\frac{1}{N}(2\epsilon + \ln\frac{6 (2N)^{50} }{0.05} )} \\
		\epsilon &\leq 0.632456 \sqrt{\epsilon+59.9584} \\
		\epsilon &\leq 5.10136
		\end{align*}
	\subparagraph{[d]} 
		\begin{align*}
		\epsilon &\leq \sqrt{\frac{1}{2N} (4\epsilon(1 +\epsilon) + \ln\frac{4 (N^2)^{50} }{0.05} ) } \\
		\epsilon &\leq 0.632456 \sqrt{\epsilon^2+\epsilon+41.3315} \\
		\epsilon &\leq 5.59313
		\end{align*}

%%%%%%%%%%%%%%%%%%%%%%%%%%%%%%%%%%%%%%%%%%%%%%%%%%%%%%%%%%%%%%%%%%%%%%%%%%%%%%%%
\section*{Bias and Variance}

\paragraph{4.} E. None of the above

	The average slope $b$ of the chosen function $g$, averaging over one million times, was 1.42. See the code attached for an experimental solution.

\paragraph{5.} B. 0.3

	The average bias of the chosen function $g$ was 0.262. See the code attached for an experimental solution.

\paragraph{6.} A. 0.2

	The variance of the chosen function $g$, averaging over one million times, was 0.237. See the code attached for an experimental solution.

\paragraph{7.} D or E.

	\subparagraph{[a]} $h(x) = b$

		$E_{out} = \text{bias } + \text{ variance} = 0.5 + 0.25 = 0.75$
	\subparagraph{[b]} $h(x) = ax$

		$E_{out} = \text{bias } + \text{ variance} = 0.5 + 0.25 = 0.75$
	\subparagraph{[c]} $h(x) = ax+b$

		$E_{out} = \text{bias } + \text{ variance} = 0.2 + 1.7 = 1.9$
	\subparagraph{[d]} $h(x) = ax^2$

		$E_{out} = \text{bias } + \text{ variance} = $
	\subparagraph{[e]} $h(x) = ax^2+b$

		$E_{out} = \text{bias } + \text{ variance} =$

%%%%%%%%%%%%%%%%%%%%%%%%%%%%%%%%%%%%%%%%%%%%%%%%%%%%%%%%%%%%%%%%%%%%%%%%%%%%%%%%
\section*{VC Dimension}

\paragraph{8.} C. $q$

	\begin{align*}
	m_\mathcal{H}(N+1) &= m_\mathcal{H}(N) - \binom{N-1}{q} \\
	m_\mathcal{H}(1 \leq q) &= 2 = 2^1 \\
	m_\mathcal{H}(2 \leq q) &= 2\cdot2 = 2^2 \\
	m_\mathcal{H}(3 \leq q) &= 2\cdot2\cdot2 = 2^3 \\
	m_\mathcal{H}(p < q) &= 2^p \\
	m_\mathcal{H}(p = q) &= 2^p - \binom{p-1}{q} = 2^p - 0 = 2^p \\
	m_\mathcal{H}(p = q+1) &= 2^p - \binom{q}{q} = 2^p - 1 \\
	\end{align*}

	Thus, at $q+1$, the growth function returns a value less than $2^{q+1}$. Therefore, the breakpoint $k=q+1$ and the VC dimension $d_{vc} = q $.

\paragraph{9.} B.

	In the case that $\bigcap_{k=1}^K \mathcal H_k = \emptyset$, then $d_{vc}(\bigcap_{k=1}^K \mathcal H_k) = 0$. Thus, the lower bound of 0 is valid. However, if $\min\{d_{vc}(\mathcal H_k)\}_{k=1}^K$ is not zero as well, then it is an invalid lower bound in this scenario. It is not a valid lower bound because there is no reason that $d_{vc}$ for some $\mathcal H_k$ in the set is necessarily zero. Thus, we can eliminate choices \textbf{[d]} and \textbf{[e]} for invalid lower bounds.

	In the case that $\bigcap_{k=1}^K \mathcal H_k$ is such that all the hypotheses amongst the sets are in common, then each $\mathcal H_k$ in the set of all $\mathcal H$ is equivalent. Then, $d_{vc}(\bigcap_{k=1}^K \mathcal H_k) = d_{vc} (\mathcal H)$, where $\mathcal H = \mathcal H_k, \forall k$. In this case, $$d_{vc} (\mathcal H) = \min\{d_{vc}(\mathcal H_k)\}_{k=1}^K = \max\{d_{vc}(\mathcal H_k)\}_{k=1}^K \leq \sum_{k=1}^K d_{vc}(\mathcal H_k).$$Thus, we can eliminate \textbf{[a]} for being a looser fit on the upper bound. 

	We can also eliminate \textbf{[c]} for being a looser fit than \textbf{[b]}. Since $d_{vc}(\bigcap_{k=1}^K \mathcal H_k)$ returns the greatest subset of all of the $\mathcal H$, the minimum function is valid. For example, if $\mathcal H_1 \subset \mathcal H_2$, then $\mathcal H_1 \cap \mathcal H_2 = \mathcal H_1$. Then, $d_{vc}(\mathcal H_1)$ is the proper $d_{vc}$. However, since $\mathcal H_1 \cap \mathcal H_2 = \mathcal H_1$, then $d_{vc}(\mathcal H_1) \leq d_{vc}(\mathcal H_2)$.

	\subparagraph{[a]} 
		$0 \leq d_{vc}(\bigcap_{k=1}^K \mathcal H_k) \leq \sum_{k=1}^K d_{vc}(\mathcal H_k)$
		
		The lower bound of 0 is valid. The upper bound of $\sum_{k=1}^K d_{vc}(\mathcal H_k)$ is valid but looser than $\min\{d_{vc}(\mathcal H_k)\}_{k=1}^K$ and $\max\{d_{vc}(\mathcal H_k)\}_{k=1}^K$.
	
	\subparagraph{[b]} 
		$0 \leq d_{vc}(\bigcap_{k=1}^K \mathcal H_k) \leq \min\{d_{vc}(\mathcal H_k)\}_{k=1}^K$
		
		The lower bound of 0 is valid. 
	
	\subparagraph{[c]} 
		$0 \leq d_{vc}(\bigcap_{k=1}^K \mathcal H_k) \leq \max\{d_{vc}(\mathcal H_k)\}_{k=1}^K$
		
		The lower bound of 0 is valid. 
	
	\subparagraph{[d]} 
		$\min\{d_{vc}(\mathcal H_k)\}_{k=1}^K \leq d_{vc}(\bigcap_{k=1}^K \mathcal H_k) \leq \max\{d_{vc}(\mathcal H_k)\}_{k=1}^K$
		
		The lower bound of $\min\{d_{vc}(\mathcal H_k)\}_{k=1}^K$ is invalid. 
	
	\subparagraph{[e]} 
		$\min\{d_{vc}(\mathcal H_k)\}_{k=1}^K \leq d_{vc}(\bigcap_{k=1}^K \mathcal H_k) \leq \sum_{k=1}^K d_{vc}(\mathcal H_k)$
		
		The lower bound of $\min\{d_{vc}(\mathcal H_k)\}_{k=1}^K$ is invalid. 

\paragraph{10.} E.
	
	Given some set of $\mathcal H$, each $\mathcal H_k$ has some $d_{vc}$. Then, if we take the intersection of $\mathcal H_\cup = \bigcup_{k=1}^K \mathcal H_k$, the super-set $\mathcal H_\cup$ contains all hypotheses that led to each $d_{vc}$ before. Thus, a lower bound of $\max\{d_{vc}(\mathcal H_k)\}_{k=1}^K$ is valid. 

	$\sum_{k=1}^K d_{vc}(\mathcal H_k)$ is not a valid upper bound. Thus, we can eliminate \textbf{[a]}, \textbf{[c]}, and \textbf{[d]}. For a counter-example of its validity, consider the hypothesis sets $\mathcal H_1$ and $\mathcal H_2$, given below, for an input space composed of three labels $[\pm, \pm, \pm]$. For $\mathcal H_1$, $d_{vc}=1$ because [-, -] cannot be shattered. For $\mathcal H_2$, $d_{vc}=1$ because [+, +] cannot be shattered. However, if we take $\mathcal H_1 \bigcap \mathcal H_2$, as $\bigcup_{k=1}^K \mathcal H_k$ does, then the $d_{vc}(\mathcal H_1 \bigcap \mathcal H_2) = 3 > d_{vc}(\mathcal H_1 ) + d_{vc}(\mathcal H_2)$. Thus, $\sum_{k=1}^K d_{vc}(\mathcal H_k)$ is not a valid upper bound. Then, the only other option $ K - 1 + \sum_{k=1}^K d_{vc}(\mathcal H_k)$ must be a valid upper bound.

	\subparagraph{[a]}
		$0 \leq d_{vc}(\bigcup_{k=1}^K \mathcal H_k) \leq \sum_{k=1}^K d_{vc}(\mathcal H_k)$

		The upper bound is invalid. 
	
	\subparagraph{[b]}
		$0 \leq d_{vc}(\bigcup_{k=1}^K \mathcal H_k) \leq K - 1 + \sum_{k=1}^K d_{vc}(\mathcal H_k)$

		The upper bound is valid. The lower bound is valid, but more loose than $\max\{d_{vc}(\mathcal H_k)\}_{k=1}^K$.
	
	\subparagraph{[c]}
		$\min\{d_{vc}(\mathcal H_k)\}_{k=1}^K \leq d_{vc}(\bigcup_{k=1}^K \mathcal H_k) \leq \sum_{k=1}^K d_{vc}(\mathcal H_k)$

		The upper bound is invalid.
	
	\subparagraph{[d]}
		$\max\{d_{vc}(\mathcal H_k)\}_{k=1}^K \leq d_{vc}(\bigcup_{k=1}^K \mathcal H_k) \leq \sum_{k=1}^K d_{vc}(\mathcal H_k)$

		The upper bound is invalid.
	
	\subparagraph{[e]}
		$\max\{d_{vc}(\mathcal H_k)\}_{k=1}^K\leq d_{vc}(\bigcup_{k=1}^K \mathcal H_k) \leq K - 1 + \sum_{k=1}^K d_{vc}(\mathcal H_k)$

		The upper bound is valid. The lower bound is valid, and more tight than 0.


\end{document}

