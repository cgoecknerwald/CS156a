% Claire Goeckner-Wald

\documentclass[10pt,letter]{article}
\usepackage{amsmath}
\usepackage{amssymb}
\usepackage{graphicx}
\usepackage{setspace}
\onehalfspacing
\usepackage{fullpage}

\begin{document}
\title{Problem Set 6}
\author{Claire Goeckner-Wald}
\maketitle 

%%%%%%%%%%%%%%%%%%%%%%%%%%%%%%%%%%%%%%%%%%%%%%%%%%%%%%%%%%%%%%%%%%%%%%%%%%%%%%%%
\section*{Overfitting and Deterministic Noise}

\paragraph{1.} \textbf{[b]} In general, deterministic noise will increase

	Assume that $\mathcal H' \subseteq \mathcal H$ is a subset such that there are fewer hypotheses in $\mathcal H'$ than in $\mathcal H$. Consider natural subsets, such as $\mathcal H_2 \subset \mathcal H_10$. Bias, or deterministic noise, is the average squared error over the $\mathcal X$-space of our average hypothesis $\overline g(x)$ from target function $f$. Because we are constricting our hypothesis set to fewer total hypotheses, we in general increase the deterministic noise, because the average hypothesis $\overline g(x)$ may move away from the target function. 

%%%%%%%%%%%%%%%%%%%%%%%%%%%%%%%%%%%%%%%%%%%%%%%%%%%%%%%%%%%%%%%%%%%%%%%%%%%%%%%%
\section*{Regularization with Weight Decay}

\paragraph{2.} \textbf{[a]} 0.03 and 0.08

	See the attached code. 

\paragraph{3.} \textbf{[d]} 0.03 and 0.08

	See the attached code. 

\paragraph{4.} \textbf{[e]} 0.04 and 0.04

	See the attached code. 

\paragraph{5.} \textbf{[e]} -1

	See the attached code. 

\paragraph{6.} \textbf{[b]} 0.06

	Between $k$ = -1000 and $k$ = 100, the least value of the out-sample error occurs at $k$ = -1, and is 0.056. See the attached code.

%%%%%%%%%%%%%%%%%%%%%%%%%%%%%%%%%%%%%%%%%%%%%%%%%%%%%%%%%%%%%%%%%%%%%%%%%%%%%%%%
\section*{Regularization for Polynomials}

\paragraph{7.} \textbf{[c]} $\mathcal H(10,0,3) \cap \mathcal H(10,0,4) = \mathcal H_2$

	$\mathcal H_n$ is the set of all polynomials of order $n$ or less. 

	Then, some hypothesis set $\mathcal H(Q,C=0,Q_0)$ would create a set of polynomials of order $\min(Q, Q_0-1)$ or less. Because $w_q = C = 0$ for $q \geq Q_0$ (versus $q > Q_0$), we use $Q_0-1$ as a potential upper bound of the orders of polynomials in the set. However, $Q$ itself is a potential upper bound the set is created using $\mathcal H_Q$ and filtering out undesirables. 

\subparagraph{[a]} $\mathcal H(10,0,3) \cup \mathcal H(10,0,4) = \mathcal H_4$

	This set is the union of: 1) a set of all polynomials of order 2 or less, 2) a set of all polynomials of order 3 or less. Thus, because this is a union, this is equivalent to the set $\mathcal H_3$, not the set $\mathcal H_4$.

\subparagraph{[b]} $\mathcal H(10,1,3) \cup \mathcal H(10,1,4) = \mathcal H_3$

	Since this is a union, we can simply consider $\mathcal H(10,1,4)$. Here, $w_q = 1$ for $4 \leq q \leq 10$. Thus, this set contains some polynomials of order greater than 3. Since $\mathcal H_3$ is limited to polynomials of order 3 and less, this is not $\mathcal H_3$.

\subparagraph{[c]} $\mathcal H(10,0,3) \cap \mathcal H(10,0,4) = \mathcal H_2$

	This set is the intersection of: 1) a set of all polynomials of order 2 or less, 2) a set of all polynomials of order 3 or less. Thus, because this is an intersection, this is equivalent to the set $\mathcal H_2$.

\subparagraph{[d]} $\mathcal H(10,1,3) \cap \mathcal H(10,1,4) = \mathcal H_1$

	The intersection of both sets contains $\mathcal H_2$ because $w_q$ is unlimited for $q < \min(3,4)$. Thus, this set contains some polynomials of order greater than 1. Since $\mathcal H_1$ is limited to polynomials of order 1 and less, this is not equivalent to $\mathcal H_1$.

%%%%%%%%%%%%%%%%%%%%%%%%%%%%%%%%%%%%%%%%%%%%%%%%%%%%%%%%%%%%%%%%%%%%%%%%%%%%%%%%
\section*{Neural Networks}

\paragraph{8.} \textbf{[d]} 45

	Consider how much each operation adds to the total. The number of weights in this situation is $6\cdot(4-1)+4\cdot(1-0) = 22$. We will use $o(\text{operation})$ as a function that returns the number of operations occurred. 

	\textbf{Forward computation: } Compute all $x_j^{(l)}$, starting with $l = 1$. Thus, the number of operations necessary here is equivalent to the number of weights. 
	\begin{align*}
	o(w_{ij}^{(l)}x_i^{(l-1)}) = 22
	\end{align*}

	\textbf{Backward computation: } For each level, we must calculate the $\delta$ for each node. Consider that for $l = 0$ and $l = L$, the number of operations here are 0. This is because for $l=0$, no node takes in a signal. For $l = L$, we do not need to do any operations as we already know the $\delta$. For each $\delta_i^{(l-1)}$, we only need $d^{(l)}$ operations. Thus, for $l = 1$ to $l = L-1$, we take $\sum_{i=0}^{d^{(l)}} d^{(l+1)}$. Here, we only need to calculate the sum for the middle layer, $l = 1$, for which $d^{(1)} = 3$. 
	\begin{align*}
	o(w_{ij}^{(l)}\delta_j^{(l)}) &= \sum_{i=1}^{d^{(1)}} d^{(1+1)} \\
	&= \sum_{i=1}^{3} 1 \\
	&= 1 + 1 + 1 = 3 \\
	\end{align*}

	\textbf{Updating the weights:} Each weight is updated once with the operation $w_{ij}^{(l)} = w_{ij}^{(l)} + \eta(x_i^{(l-1)}\delta_j^{(l)})$. Therefore, the number of operations for this calculation is equivalent to the number of weights. 
	\begin{align*}
	o(x_i^{(l-1)}\delta_j^{(l)}) = 22
	\end{align*}

	Totaling these up, we get $22+22+3 = 47$ operations. 47 is closer to 45 than to 50, thus the answer choice.

\paragraph{9.} \textbf{[a]} 46
	
	For all levels except for the last, two levels $l$ and $l+1$ with $x^{(l)}$ and $x^{(l+1)}$ units respectively, require $x^{(l)}\cdot(x^{(l+1)}-1)$ weights to be fully connected. The last level connection requires $(x^{(l)}\cdot x^{(l+1)})$ weights because there is no bias term in the output. We can minimize the number of weights necessary by creating the hidden layers such that each layer $l$ has a minimal number of units, $x^{(l)}$. Thus, we take $x^{(l)} = 2$ for each hidden layer, because we require a $x_0$, and at least one more term to be fully connected. Thus, the minimum number of weights $o_{\text{min}}$ is as follows.

	From the input to the first hidden layer:
	\begin{align*}
	o_{\text{min}} &= 10(2 - 1) = 10 
	\end{align*}
	Between hidden layers, where terms $(x_0^{(l)}, x_1^{(l)})$ fully connect to $(x_0^{(l+1)}, x_1^{(l+1)})$, recalling that the term $x_0^{(l+1)}$ should not be connected to the previous layer:
	\begin{align*}
	o_{\text{min}} &= 2(2 - 1) = 2
	\end{align*}
	There are 36 hidden units. We desire that each layer is composed of $(x_0, x_1)$. Thus, there must be $36/2 = 18$ hidden layers. Then, there are $18 - 1 = 17$ connection spaces between these 18 layers. Each connection space requires $o_{\text{min}} = 2$ weights for full connection to the previous layer. (Note that the first and last hidden layers are not necessarily the same.)
	\begin{align*}
	2 \cdot (17 \text{ layers}) = 34 
	\end{align*}
	From the last hidden layer to the output (since there is only one term in the output):
	\begin{align*}
	o_{\text{min}} &= 2(1) = 2 
	\end{align*}

	Then, our minimum number of weights is $34 + 10 +2= 46$. Using this same logic, if we instead had 9 hidden layers of 4 nodes each, the number of weights would be $10*(4-1) + 8*(4*(4-1)) + 4*(1) = 130$. Thus, 46 is the likely minimum. 

	Note: the formula for number of weights $o$, given $\lambda$ hidden layers of equal numbers of units each (assuming that $\frac{36}{\lambda} \in \mathbb Z$) is $o(\lambda) = 10(\frac{36}{\lambda} - 1) + (\lambda - 1)(\frac{36}{\lambda}\cdot (\frac{36}{\lambda} - 1)) + \frac{36}{\lambda} \cdot 1$. We desire to find the minimum and maximum of this function where $\lambda \in \mathbb Z$. 

\paragraph{10.} \textbf{[c]} 494
	
	Follow from the intuition developed in the first problem. Consider a single hidden layer. Then, it must be composed of 36 units, or 36 nodes. Thus, we can calculate the maximum number of weights, $o_{\text{max}}$.

	From the input to the hidden layer of 36 nodes:
	\begin{align*}
	o_{\text{max}} &= 10(36 - 1) = 350
	\end{align*}
	From the hidden layer of 36 nodes to the output:
	\begin{align*}
	o_{\text{max}} &= 36(1) = 36
	\end{align*}

	Then, our maximum number of weights is $350 + 36 = 386$.

	However, consider two hidden layers, each of 18 units, or 18 nodes.

	From the input to the hidden layer of 36 nodes:
	\begin{align*}
	o_{\text{max}} &= 10(18 - 1) = 170
	\end{align*}
	Between the two hidden layers:
	\begin{align*}
	o_{\text{max}} &= 18(18 - 1) = 306
	\end{align*}
	From the last hidden layer of 18 nodes to the output:
	\begin{align*}
	o_{\text{max}} &= 18(1) = 18
	\end{align*}

	Then, our maximum number of weights is $170 + 306 + 18 = 494$.

	Consider three hidden layers, each of $12$ units, or nodes.

	From the input to the hidden layer of 36 nodes:
	\begin{align*}
	o_{\text{max}} &= 10(12 - 1) = 110
	\end{align*}
	Between the three hidden layers:
	\begin{align*}
	o_{\text{max}} &= 12*(12-1) \cdot (2 \text{ levels}) = 264
	\end{align*}
	From the last hidden layer of 18 nodes to the output:
	\begin{align*}
	o_{\text{max}} &= 12(1) = 12
	\end{align*}

	Then, our maximum number of weights is $170 + 306 + 18 = 386$. Thus, our likely maximum number of weights occurs at 494.

\end{document}

