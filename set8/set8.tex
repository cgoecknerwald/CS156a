% Claire Goeckner-Wald

\documentclass[10pt,letter]{article}
\usepackage{amsmath}
\usepackage{amssymb}
\usepackage{bm}
\usepackage{graphicx}
\usepackage{setspace}
\onehalfspacing
\usepackage{fullpage}

\begin{document}
\title{Problem Set 8}
\author{Claire Goeckner-Wald}
\maketitle 

%%%%%%%%%%%%%%%%%%%%%%%%%%%%%%%%%%%%%%%%%%%%%%%%%%%%%%%%%%%%%%%%%%%%%%%%%%%%%%%%
\section*{Primal versus Dual Problem}

\paragraph{1.} [d] a quadratic programming problem with $d+1$ variables

	In the primal problem, we aim to minimize $\frac{1}{2} \bm{w^T}\bm{w}$ with the constraint $y_n(\bm{w^T}x_n +b) \geq 1$ for $n = 1, 2, \dots N$. Because $y_n$ and $x_n$ are given as input data, our variables are $\bm{w}$ and $b$. Note that in this scenario, $\bm{w} = (w_1, w_2, \dots w_d)$, and $b = w_0$. Thus, we have $d+1$ variables. 

\section*{Polynomial Kernels}

\paragraph{2.} [a] 0 versus all

	See attached code. 

\paragraph{3.} [a] 1 versus all

	See attached code. 

\paragraph{4.} [c] 1800

	See attached code. 

\paragraph{5.} [d] Maximum $C$ achieves the lowest $E_in$

	See attached code. 

\paragraph{6.} [b] When $C = 0.001$, the number of support vectors is lower at $Q = 5$.

	See attached code. 

\section*{Cross Validation}

\paragraph{7.} [b] $C = 0.001$ is selected most often.

	See attached code. 

\paragraph{8.} [c] $0.005$

	See attached code. 

\section*{RBF Kernel}

\paragraph{9.} [e] $C = 10^6$

	See attached code. 

\paragraph{10.} [c] $C = 100$

	See attached code. 

\end{document}

