% Claire Goeckner-Wald

\documentclass[10pt,letter]{article}
\usepackage{amsmath}
\usepackage{amssymb}
\usepackage{graphicx}
\usepackage{setspace}
\onehalfspacing
\usepackage{fullpage}
\begin{document}

\title{Set 3}
\author{Claire Goeckner-Wald}
\maketitle 

%%%%%%%%%%%%%%%%%%%%%%%%%%%%%%%%%%%%%%%%%%%%%%%%%%%%%%%%%%%%%%%%%%%%%%%%%%%%%%%%
\section*{Generalization Error}

\paragraph{1)} A. 500

	\begin{align*}
	0.03 &\geq 2 M e ^{-2(\epsilon)^2N} \\
	0.03 &\geq 2 (1) e ^{-2(0.05)^2N} \\
	0.03 &\geq 2 e ^{(-.005)N} \\
	\frac{0.03}{2} &\geq e ^{(-.005)N} \\
	\ln (\frac{0.03}{2}) &\geq (-.005)N \\
	\frac{\ln (\frac{0.03}{2})}{-.005} &\geq N \\
	\frac{\ln (\frac{0.03}{2})}{-.005} &\approx 840 \\
	\end{align*}

\paragraph{2)} B. 1000

	\begin{align*}
	0.03 &\geq 2 M e ^{-2(\epsilon)^2N} \\
	0.03 &\geq 2 (10) e ^{-2(0.05)^2N} \\
	0.03 &\geq 20 e ^{(-.005)N} \\
	\frac{0.03}{20} &\geq e ^{(-.005)N} \\
	\ln (\frac{0.03}{20}) &\geq (-.005)N \\
	\frac{\ln (\frac{0.03}{20})}{-.005} &\geq N \\
	\frac{\ln (\frac{0.03}{20})}{-.005} &\approx 1300 \\
	\end{align*}


\paragraph{3)} C. 1500

	\begin{align*}
	0.03 &\geq 2 M e ^{-2(\epsilon)^2N} \\
	0.03 &\geq 2 (100) e ^{-2(0.05)^2N} \\
	0.03 &\geq 200 e ^{(-.005)N} \\
	\frac{0.03}{200} &\geq e ^{(-.005)N} \\
	\ln (\frac{0.03}{200}) &\geq (-.005)N \\
	\frac{\ln (\frac{0.03}{200})}{-.005} &\geq N \\
	\frac{\ln (\frac{0.03}{200})}{-.005} &\approx 1761 \\
	\end{align*}

%%%%%%%%%%%%%%%%%%%%%%%%%%%%%%%%%%%%%%%%%%%%%%%%%%%%%%%%%%%%%%%%%%%%%%%%%%%%%%%%
\section*{Break Point}

\paragraph{4)} B. 5
	
	Since the $d_{vc}$ is 1 less than the break point, and since $d_{vc} = d + 1$, then for $\mathbb R^3$, then $d = 3$. Thus, the break point is 5.
	
%%%%%%%%%%%%%%%%%%%%%%%%%%%%%%%%%%%%%%%%%%%%%%%%%%%%%%%%%%%%%%%%%%%%%%%%%%%%%%%%
\section*{Growth Function}

\paragraph{5)} 

	The growth function is polynomial in the case that the hypothesis set has a break point. Otherwise, it is $2^N$.

	\subparagraph{i.} $1 + N$

		As shown in lecture 5, example 1, the growth function is $1 + N$.

	\subparagraph{ii.} $1 + N + \binom{N}{2}$

	\subparagraph{iii.} $\sum_{i=1}^{\lfloor{\sqrt N} \rfloor} \binom{N}{i}$

	\subparagraph{iv.} $2^{\lfloor{N/2}\rfloor}$

		This function is neither polynomial nor $2^N$. Therefore, this is not a possible growth function.

	\subparagraph{v.} $2^N$

		As shown in lecture 5, example 3, the convex set in $\mathbb R^2$ shatters $N$ points. Thus, the growth function is $2^N$.

%%%%%%%%%%%%%%%%%%%%%%%%%%%%%%%%%%%%%%%%%%%%%%%%%%%%%%%%%%%%%%%%%%%%%%%%%%%%%%%%
\section*{Fun with Intervals}

\paragraph{6)}

\paragraph{7)}

\paragraph{8)}

%%%%%%%%%%%%%%%%%%%%%%%%%%%%%%%%%%%%%%%%%%%%%%%%%%%%%%%%%%%%%%%%%%%%%%%%%%%%%%%%
\section*{The Triangle}

\paragraph{9)}

%%%%%%%%%%%%%%%%%%%%%%%%%%%%%%%%%%%%%%%%%%%%%%%%%%%%%%%%%%%%%%%%%%%%%%%%%%%%%%%%
\section*{Non-Convex Sets: Concentric Circles}

\paragraph{10)}

\end{document}
